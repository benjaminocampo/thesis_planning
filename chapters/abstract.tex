%Two resources useful for abstract writing. Guidance of how to write an
% abstract/summary provided by Nature:
% https://cbs.umn.edu/sites/cbs.umn.edu/files/public/downloads/Annotated_Nature_abstract.pdf
% https://writingcenter.gmu.edu/guides/writing-an-abstract
\chapter*{Resumen}

Los planificadores en planning clásico encuentran planes con éxito aún para
tareas realmente complejas. Para esto, la mayoría de planificadores realizan una
etapa de preprocesamiento sobre la especificación de la tarea para obtener una
representación de mucho más bajo nivel de abstracción. Este proceso es conocido
como proceso de grounding. Cada vez que la tarea instanciada es demasiado grande
para ser generada, la tarea no puede ser resuelta por el planificador. En esta
tesis proponemos un proceso alternativo, denominado grounding heurístico, que
guía el proceso de grounding, instanciando aquellas partes de la tarea que son
relevantes. Para ello, se trabajó sobre modelos de aprendizaje supervisado,
planes relajados y codificaciones ad-hoc y por word embeddings.

\noindent
\textbf{Palabras claves:} Planning, aprendizaje automático, proceso de
grounding, codificación one-hot, word embeddings.

\section*{Abstract}

Planners in classical planning are successful in finding plans, even for complex
tasks. To do so, most planners rely on a preprocessing stage that computes a
grounded representation of the task. This process is known as the grounding
process. However, if the grounded task is too big to be generated, it can not be
tackled by the planner. In this thesis, we propose an alternative approach
called heuristic grounding. This method, guides the grounding process,
instantiating only the parts of the task that are relevant using machine
learning techniques, relaxed plans, and ad-hoc and word embeddings encoders.

\noindent
\textbf{Keywords:} Planning, machine learning, grounding process, one-hot
encoding, word embeddings.