%Two resources useful for abstract writing.
% Guidance of how to write an abstract/summary provided by Nature:
% https://cbs.umn.edu/sites/cbs.umn.edu/files/public/downloads/Annotated_Nature_abstract.pdf
% https://writingcenter.gmu.edu/guides/writing-an-abstract
\chapter*{Resumen}

La planificación automática, o simplemente \emph{Planning}, es una de las área
más antiguas y centrales de la Inteligencia Artificial que estudia la generación
automática de acciones por parte de agentes inteligentes con el proposito de
alcanzar una meta a partir de una situación inicial. Esta secuencia de acciones
es denominada un \emph{plan} de la tarea. Los planificadores son los encargados
de encontrar un plan a partir de una especificación de la tarea a resolver. Para
esto, la mayoría de planificadores realizan una etapa de preprocesamiento sobre
la especificación para obtener una representación de mucho más bajo
nivel de abstracción. Este proceso se conoce como \emph{proceso de grounding},
el cual se realiza previamente a la búsqueda de un plan. Para
aquellas tareas en las que el proceso de grounding no puede realizarse debido al
agotamiento de recursos computacionales, el planificador no puede empezar la
búsqueda de una solución al no poder generar la tarea groundeada. Por otro lado,
la búsqueda que realizan los planificadores es guiada a partir de una función
heurística que se obtiene a partir de información de la tarea. En particular, el
sistema Fast Downward (FD), el framework para resolver tareas de planning más famoso
y utilizado por la comunidad de planning, implementa una búsqueda guiada a
partir de planes de una tarea simplificada. Estas soluciones son denominadas
\emph{planes relajados}. Dado que los planes relajados han mostrado contener la
información necesaria para guiar el proceso de búsqueda, entonces es posible que
también puedan guiar el proceso de grounding. En este trabajo mostramos un
proceso alternativo denominado \emph{grounding heurístico} que consiste en
generar solamente una proyección de la tarea cuando grounding total no es
factible. Este proceso es un mecanismo guiado en donde identificamos las
acciones que son relevantes para encontrar un plan utilizando métodos de
aprendizaje automático. Como características de los modelos se utilizaron planes
relajados en motivación al método de búsqueda de FD. Por último, se propusieron
dos codificaciones para los datos de entrada de los modelos de aprendizaje, una
codificación ad-hoc basasda en las codificaciones de tipo one-hot, y otra a
partir de word embeddings provenientes del lenguaje natural. Los resultados
obtenidos justifican el uso de planes relajados para guiar el proceso de
grounding.


\noindent % Provide your key words
\textbf{Keywords:} a maximum of five keywords/keyphrase separated by commas