%Two resources useful for abstract writing.
% Guidance of how to write an abstract/summary provided by Nature:
% https://cbs.umn.edu/sites/cbs.umn.edu/files/public/downloads/Annotated_Nature_abstract.pdf
% https://writingcenter.gmu.edu/guides/writing-an-abstract
\chapter*{Resumen}

La planificación automática, o simplemente \emph{Planning}, es una de las área
más antiguas y centrales de la Inteligencia Artificial. Estudia la generación
automática de acciones por parte de agentes inteligentes con el proposito de
resolver una tarea que especifique una meta y una situación inicial. Esta
secuencia de acciones es denominada un \emph{plan} de la tarea. Los
planificadores son los encargados de encontrar un plan a partir de una
especificación de la tarea a resolver. Para esto, la mayoría de planificadores
realizan una etapa de preprocesamiento sobre la especificación para obtener una
representación de mucho más bajo nivel de abstracción. Este proceso se conoce
como \emph{proceso de grounding}, el cual se realiza previamente a la búsqueda
de un plan. Para aquellas tareas en las que el proceso de grounding no puede
realizarse debido al agotamiento de recursos computacionales, el planificador no
puede empezar la búsqueda de una solución al no poder generar la tarea
groundeada. Por otro lado, la búsqueda que realizan los planificadores es guiada
a partir de una función heurística que se obtiene a partir de información de la
tarea. En particular, el sistema Fast Downward (FD), el framework para resolver
tareas de planning más famoso y utilizado por la comunidad de planning,
implementa una búsqueda guiada a partir de planes de una tarea simplificada.
Estas soluciones son denominadas \emph{planes relajados}. Dado que los planes
relajados han mostrado contener la información necesaria para guiar el proceso
de búsqueda, entonces es posible que también puedan guiar el proceso de
grounding. En este trabajo mostramos un proceso alternativo denominado
\emph{grounding heurístico} que consiste en generar solamente una proyección de
la tarea cuando grounding total no es factible. Este proceso es un mecanismo
guiado en donde identificamos las acciones que son relevantes para encontrar un
plan utilizando métodos de aprendizaje automático. Como características de los
modelos se utilizaron planes relajados en motivación al método de búsqueda de
FD. Por último, se propusieron dos codificaciones para los datos de entrada de
los modelos de aprendizaje, una codificación ad-hoc basada en las
codificaciones de tipo one-hot, y otra a partir de word embeddings provenientes
del lenguaje natural. Los resultados obtenidos justifican el uso de planes
relajados para guiar el proceso de grounding.


\noindent % Provide your key words
\textbf{Palabras claves:} Planning, aprendizaje automático, proceso de
grounding, modelos lineales, regresión logística, XGBoost, redes neuronales,
word embeddings.

\section*{Abstract}

Automated planning, or simply Planning, is one of the oldest and most central
areas of Artificial Intelligence.  It studies the automatic generation of
actions by part of intelligent agents to solve a task that specifies a goal and
an initial state. This sequence of actions is called a plan of the task. The
planners are in charge of finding a plan from a specification of the task. To
find a solution, most planners rely on a  preprocessing stage through the
specification to get a grounded representation of the task. This process is
known as grounding, and it needs to be done previously in search of a plan. For
those tasks in which the grounding process cannot be done due to computational
resources, the planner cannot start searching for a solution since it can't
generate the grounded task. On the other hand, the search carried out by
planners is guided by a heuristic function that is obtained from task
information. In particular, the Fast Downward (FD) system, the framework to
solve planning tasks most famous and used by the community of planning,
implements a guided search from plans of a simplified task. These solutions are
called relaxed plans. Since relaxed plans have been shown to contain the
information necessary to guide the search process, then it is also possible that
they can guide the grounding process. In this thesis, we show an alternative
approach called heuristic grounding that consists of generating only a
projection of the task when total grounding is not feasible. This process is a
guided mechanism where We identify the actions that are relevant to find a plan
using methods of machine learning. As the features of the models, relaxed plans
were used in the motivation to the heuristic search of FD. Finally, two
encodings were proposed for the input data of the learning models, ad-hoc
encoding based on one-hot encodings methods, and word embeddings from the
natural language processing. The results obtained justify the use of relaxed
plans to guide the process of grounding.

\textbf{Keywords:} Planning, machine learning, grounding process, linear models, logistic
regression, XGBoost, neural networks, word embeddings.