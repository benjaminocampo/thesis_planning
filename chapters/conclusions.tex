\chapter{Conclusiones generales}
\label{ch:con}

\section{Reflección}

La motivación de grounding heurístico surgió tras observar que gran parte de los
problemas de planning solo utilizan una pequeña fracción del total de posibles
acciones para resolver una tarea. Es común en el área trabajar sobre tareas de
con cientos de miles de acciones, pero que resultan en planes de solamente
cientos de ellas. Aun así, los planificadores para encontrar una solución de la
tarea, deben realizar una búsqueda exhaustiva sobre el total de acciones. Esta
búsqueda es guiada a partir de una función heurística definida según
información que se puede extraer de la tarea. En el caso de Fast Downward, el
framework de planning más utilizado por la comunidad para la resolución de
problemas de planning, implementa un algoritmo genérico de búsqueda cuya función
heurística se basa en el uso de planes relajados. Esto motivó el uso de planes
relajados en grounding heurístico. Si los planes relajados tienen la información
para guiar el proceso de búsqueda, entonces también pueden guiar el proceso de
grounding.

Por otro lado, el poder predictivo de los algoritmos de aprendizaje automático
ha llegado a incluso alcanzar tasas menores al error humano. Por lo que en este
trabajo se buscó investigar de que manera se podían aplicar técnicas de
aprendizaje automático que permitiesen reconocer la relación entre los planes
relajados, acciones, y planes reales. Nuestro objetivo fue encontrar un modelo
que permita estimar la probabilidad de que una acción sea relevante para algún
plan real, dado su plan relajado.

A partir de esto, surgió una etapa de pruebas de concepto donde verificábamos de
que manera disponer el material de entrenamiento y test para los modelos de
aprendizaje automático y de que forma codificar los planes relajados y las
acciones. A la par de estas pruebas de concepto, se desarrolló la arquitectura
del sistema que utilizaríamos para ejecutar futuros experimentos.

El algoritmo final estuvo constituido por una etapa de extracción de los datos a
partir del generador de tareas, su preservación de datos, selección de problemas
de entrenamiento y test, generación de ejemplos etiquetados, construcción de
ventanas de planes relajados, y codificación de características.

En la etapa de codificación fue donde pusimos nuestro mayor enfoque proponiendo
dos codificaciones, una de tipo ad-hoc inspirada por una codificación one-hot, y
otra a partir de word embeddings del preprocesamiento del lenguaje natural. Dos
métodos muy distintos pero igual de prometedores para entrenar modelos de
aprendizaje automático. Con respecto a los clasificadores utilizados, se
utilizaron los representantes de tres grandes familia de modelos. En el caso de
los modelos lineales, una regresión logística. De los algoritmos de ensemble,
XGBoost. Y del mundo del deep learning y las redes neuronales, el perceptrón
multicapa. Cada método y clasificador trabajado fueron muy distintos y con
características únicas, lo cual resultó excelente para enriquecer nuestro
espacio de búsqueda y experimentación.

Con respecto a los resultados finales, presentamos dos modelos que obtuvieron
excelentes resultados entrenados sobre acciones provenientes de los esquemas de
acción \emph{take\_image}, y \emph{calibrate}.

Para el caso de \emph{take\_image}, el mejor modelo se obtuvo sorprendentemente
con una regresión logística donde, para el conjunto de test, fue capaz de
predecir 5865 acciones relevantes para un total de 6432 de ellas. Para el caso
de las no relevantes, pudo filtrar 615 de 1877. Esto es equivalente a un 91\% de
acciones relevantes que fueron correctamente identificadas y se les asignó una
probabilidad mayor a un umbral de 0.416, y un 33\% de acciones no relevantes que
recibieron una probabilidad menor a ese umbral. Además, se pudo observar a
través de validación cruzada que el modelo se mantuvo robusto ante variantes en
los datos de entrenamiento, lo cual lo hace uno de los mejores resultados de
este trabajo.

Por otro lado, el perceptrón multicapa dio también buenos resultados con el
esquema de acción \emph{calibrate} donde las predicciones en el conjunto de test
se distribuyeron de manera distinta a los resultados en validación cruzada, se
mantuvo el orden las acciones relevantes, siguen manteniendo una probabilidad
mayor que las no relevantes. Lo cual en términos de grounding heurístico es
importante. En particular, para un umbral de 0.1 el modelo fue capaz de predecir
correctamente un 81\% de las acciones relevantes correctamente y logró filtrar
hasta un 33\% de las no relevantes. Lo cual lo convierte en otro candidato para
ser utilizado en el algoritmo de grounding heurístico.

\section{Trabajo a futuro}

Una de los experimentos principales como trabajo a futuro es agregar los modelos
obtenidos en la implementación de grounding heurístico en el framework de Fast
Downward y corroborar que tales modelos de aprendizaje, efectivamente permiten
resolver los problemas de test que un principio no son instanciables y, por
ende, no tienen solución bajo la implementación vanilla de Fast Downward.

Otro experimento posible es mantener un registro de los tiempos de predicción de
cada modelo a nivel de acción o a nivel de batch de acciones. Recordemos que
para realizar la predicción de una acción se debe primero realizar su
preprocesamiento, construir las ventanas del plan relajado, realizar la
predicción de cada ventana, y finalmente agrupar las predicciones para determinar
su probabilidad. Este preprocesamiento podría ser un cuello de botella si no es
implementado y optimizado de manera adecuada en el algoritmo de grounding
heurístico, teniendo en cuenta que además los planificadores deben encontrar la
solución del problema en un tiempo limitado. Por lo tanto, un debido análisis
del tiempo necesario para llevar a cabo el preprocesamiento y predicción de los
modelos es requerido antes de ser incluidos en el algoritmo de Fast Downward.

Para el caso de la codificación por word embeddings, se puede tomar ventaja de
la codificación por promedio normalizado desarrollada en la sección
\ref{method:wb}. Una posibilidad es utilizar distintos tamaños de ventana entre
los datos de entrenamiento y de test. Por ejemplo, mantener un tamaño de ventana
de $3$ para los planes relajados del conjunto de entrenamiento, y otro de $5$
para los planes relajados del conjunto de test. Resolviendo los problemas de
dimensiones en las matrices promediando las representaciones de las palabras que
componen cada ventana. Como los embeddings mantienen la misma dimensión de
salida para todas las palabras, entonces las matrices de entrenamiento y de test
mantendrían la misma estructura y dimensión. Este método puede ser una solución
a la explosión de ejemplares a causa de la confección de las ventanas de planes
relajados, pero puede degradar la información al considerar tamaños de ventanas
mucho mayores. Investigar el balance necesario entre calidad del modelo, y
degradación de información puede ser un posible lineamiento para
experimentaciones futuras.

En virtud de mejorar el sistema de experimentación desarrollado en el capítulo
\ref{ch:method} se puede sincronizar la plataforma de \emph{MLFlow} con la base
de datos SQL para que se mantengan registro de los modelos. Mantener
persistencia organizada de modelos en la nube. Otra opción es generar tareas en
\emph{Jenkins}, un sistema de gestión de pipelines de ejecución, que permitan
entrenar, evaluar, registrar, y almacenar los resultados de manera
independiente, sin necesidad de conocer su implementación. También se puede
optimizar algunos procesos de manera que aprovechen el hardware disponible de la
computadora con \emph{Cython}.