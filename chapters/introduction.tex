\chapter{Introducción}
\label{ch:into} % This how you label a chapter and the key (e.g., ch:into) will be used to refer this chapter ``Introduction'' later in the report. 
% the key ``ch:into'' can be used with command \ref{ch:intor} to refere this Chapter.

La planificación automática, o simplemente \emph{Planning}, es una de las áreas centrales de la inteligencia artificial debido a su extenso uso en dominios, tales como, control de misiones espaciales \citep{RabideauG-et-al-2001}, manejo de crisis \citep{Bienkowki-1995}, generación de textos narrativos \citep{Goudoulakis-et-al-2016}, o robótica \citep{Munoz-et-al-2016}.

El objetivo es definir un modelo que se asemeje a una tarea o problema de nuestro entorno por medio de una
especificación donde se describan todos los detalles en que esta consiste. Estas son el estado inicial, la meta, y un conjunto de acciones. El estado inicial describe las propiedades válidas iniciales dentro del entorno. La meta o estado final representa cuales son las propiedades deseables en él. Y el conjunto de acciones está
compuesto de transformadores de estados que permiten alterar estas propiedades. Si se obtiene una secuencia de acciones que sea aplicable en el estado inicial, y que luego de su ejecución conlleve a la meta, entonces dicha secuencia es considerada un plan del problema. \citep{Georgievski-et-al-2016}

Para hallar un plan que resuelva la tarea, se realizan técnicas de búsqueda y optimización, efectuadas por \emph{planners}, algoritmos que computan el comportamiento de un agente por medio de una descripción del problema comunmente definida en el \emph{Planning Domain Definition Language} (PDDL). En PDDL, se especifican las propiedades del entorno en términos de predicados, y las transformaciones por medio de esquemas de acción. Estas consisten en expresiones parametrizadas que pueden ser instanciadas por un conjunto de objetos. El \emph{planner} por medio de la especificación, define un espacio de búsqueda cuya dificultad de encontrar un plan aumenta con su complejidad. \citep{Georgievski-et-al-2016}

Sin embargo, la mayoría de los \emph{planners} trabajan sobre una representación sin variables libres. Por consecuente, estos computan todas las instanciaciones que asignan los objetos a los argumentos de los predicados y esquemas de acción definidas por el PDDL del problema. Este proceso, conocido como \emph{grounding}, es exponencial en la cantidad de argumentos de los esquemas de acción y predicados, llegando a obtener una cantidad inmensurable de instancias cuando el número de parámetros definidos es alto. Esto puede llevar a la falla por parte del \emph{planner} para resolver la tarea sin antes haber tenido la posibilidad de realizar la búsqueda en el espacio de instancias, incluso cuando en la práctica solo una pequeña fracción de estas instancias ocurren en los planes del problema. \citep{Gnad_Torralba_Dominguez_Areces_Bustos_2019}

Ahora bien, si se instancian las acciones necesarias para confeccionar por lo menos un plan, entonces el proceso de búsqueda encontraría alguna de tales soluciones \citep{TODO}. Por ende, surge la siguiente pregunta ¿cómo determinamos que acciones son relevantes para algún plan de tal manera que puedan ser incluidas en el proceso de \emph{grounding}? Más formalmente, ¿existe alguna función $F$ que dada una acción $A$ y un problema $P$, determine que tan relevante es $A$ para hallar un plan en $P$?

Esta pregunta es las que nos llevó a considerar el uso de técnicas de \emph{machine learning}. La idea principal fue lograr encontrar un modelo que prediga la probabilidad de que una acción sea relevante a partir de planes de varios problemas y acciones etiquetadas. No obstante, dado que las tareas que nos interesa resolver son aquellas que no pueden ser instanciadas. Utilizar problemas equivalentes para construir el material de entrenamiento no es una posibilidad, ya que no podríamos computar algún plan de la tarea ni determinar que acciones son relevantes para anotarlas. Aún en los casos en que esto sea posible, hacerlo es realmente costoso en términos computacionales y de tiempo necesario \citep{?}. Estas son algunas de las dificultades que nos llevó a utilizar una aproximación del plan real, conocida como \emph{relaxed plan} o plan relajado, y PDDL's de problemas más sencillos de resolver que permitan guiar el proceso de \emph{grounding} en otros más complejos.

Por otro lado, los modelos de \emph{machine learning} dependen fuertemente de como se representan los datos que uno tiene disponible \citep{Heaton-2016-AnEA}. Para obtener un modelo de aprendizaje supervisado se necesita construir un vector de \emph{features} que permita codificar nuestros datos, en particular, un plan relajado y una acción. WIP (No hablar de Word Embeddings solo de la codificación custom) Una posibilidad es utilizar el concepto de \emph{word embeddings} proveniente del procesamiento del lenguaje natural. Estos tienen como objetivo proyectar palabras y oraciones a un espacio $N$ dimensional que preserve la semántica de las palabras, es decir, expresiones con un significado similar, son dispuestas cerca en el espacio. \citep{Mikolov-Ilya-Kai-Greg-Jeffrey-2013, Pennington-Jeffrey-Socher-Richard-Manning-Christopher-2014, Bojanowski-Grave-Joulin-Mikolov-2016}.
La intuición principal es obtener un modelo de lenguaje, pensando un plan como una oración, que caracterice el lenguaje generado por los planes relajados y las acciones instanciadas.

En resumen, la hipótesis inicial de este trabajo es investigar distintas codificaciones a partir de \emph{word embeddings} que permitan al modelo de \emph{machine learning} reconocer la relación entre los planes relajados, acciones instanciadas, y los planes reales, de tal manera que guíen el proceso de \emph{grounding}.