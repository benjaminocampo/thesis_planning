\chapter{Introducción}
\label{ch:into} % This how you label a chapter and the key (e.g., ch:into) will be used to refer this chapter ``Introduction'' later in the report. 
% the key ``ch:into'' can be used with command \ref{ch:intor} to refere this Chapter.

La planificación automática, o simplemente \emph{Planning}, es una de las áreas centrales del aprendizaje automático debido a su extenso uso en dominios tales como control de misiones espaciales \citep{RabideauG-et-al-2001}, manejo de crisis \citep{Bienkowki-1995}, generación de textos narrativos \citep{Goudoulakis-et-al-2016}, o robótica \citep{Munoz-et-al-2016}.

El objetivo es definir un modelo que se asemeje a una tarea o problema de nuestro entorno por medio de una
especificación donde se describan todos los detalles en que esta consiste. Estas son el estado inicial, la meta, y un conjunto de acciones. El estado inicial describe las propiedades válidas iniciales dentro del entorno. La meta o estado final representa cuales son las propiedades deseables en el. Y el conjunto de acciones está
compuesto de transformadores de estados que permiten alterar estas propiedades. Si se obtiene una secuencia de acciones que sea aplicable en el estado inicial, y que luego de su ejecución conlleve a la meta, entonces dicha secuencia es considerada un plan del problema. \citep{Georgievski-et-al-2016}

Para hallar un plan que resuelva la tarea, se realizan técnicas de búsqueda y optimización efectuadas por \emph{planners} que reciben como entrada este modelo comúnmente definido en el \emph{Planning Domain Definition Language} (PDDL) en términos de predicados, para las propiedades del entorno, y esquemas de acción, para las transformaciones. Estas consisten en expresiones parametrizadas que pueden ser instanciadas con un conjunto de objetos. El \emph{planner} por medio de la especificación, define un espacio de búsqueda cuya dificultad de encontrar un plan aumenta con la complejidad del espacio. \citep{Georgievski-et-al-2016}

Sin embargo, la mayoría de los \emph{planners} trabajan sobre una representación sin variables libres. Por ende, estos computan todas las instanciaciones que asignan los objetos a los argumentos de los predicados y esquemas de acción definidas por el PDDL del problema. Este proceso, conocido como \emph{grounding}, es exponencial en la cantidad de argumentos de los esquemas de acción y predicados, llegando a obtener una cantidad inmensurable de instancias cuando sus \emph{signatures} o la cantidad de objetos del problema, son altas. Esto puede llevar a la falla por parte del \emph{planner} para resolver la tarea sin antes tenido la posibilidad de realizar la búsqueda en el espacio de instancias, incluso cuando en la práctica solo una pequeña fracción de estas instancias ocurren en los planes del problema. \citep{Gnad_Torralba_Dominguez_Areces_Bustos_2019}

WIP
%%%%%%%%%%%%%%%%%%%%%%%%%%%%%%%%%%%%%%%%%%%%%%%%%%%%%%%%%%%%%%%%%%%%%%%%%%%%%%%%%%%
\section{Background}
\label{sec:into_back}

%%%%%%%%%%%%%%%%%%%%%%%%%%%%%%%%%%%%%%%%%%%%%%%%%%%%%%%%%%%%%%%%%%%%%%%%%%%%%%%%%%%
\section{Problem statement}
\label{sec:intro_prob_art}

%%%%%%%%%%%%%%%%%%%%%%%%%%%%%%%%%%%%%%%%%%%%%%%%%%%%%%%%%%%%%%%%%%%%%%%%%%%%%%%%%%%
\section{Aims and objectives}
\label{sec:intro_aims_obj}

%%%%%%%%%%%%%%%%%%%%%%%%%%%%%%%%%%%%%%%%%%%%%%%%%%%%%%%%%%%%%%%%%%%%%%%%%%%%%%%%%%%
\section{Solution approach}
\label{sec:intro_sol} % label of Org section


\subsection{A subsection 1}
\label{sec:intro_some_sub1}


\subsection{A subsection 2}
\label{sec:intro_some_sub2}

\subsubsection{A subsection 1 of a subsection}
\label{sec:intro_some_subsub1}

\subsubsection{A subsection 2 of a subsection}
\label{sec:intro_some_subsub2}

%%%%%%%%%%%%%%%%%%%%%%%%%%%%%%%%%%%%%%%%%%%%%%%%%%%%%%%%%%%%%%%%%%%%%%%%%%%%%%%%%%%
\section{Summary of contributions and achievements} %  use this section 
\label{sec:intro_sum_results} % label of summary of results


%%%%%%%%%%%%%%%%%%%%%%%%%%%%%%%%%%%%%%%%%%%%%%%%%%%%%%%%%%%%%%%%%%%%%%%%%%%%%%%%%%%
\section{Organization of the report} %  use this section
\label{sec:intro_org} % label of Org section

