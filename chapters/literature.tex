\chapter{Fundamentos Teóricos}
\label{ch:lit_rev}

En esta sección se detallaran una serie de definiciones teóricas, necesarias
para comprender el enfoque de esta tesis. Partiendo desde el área de
\emph{planning}, detallando más concretamente que son los estados y acciones, el
lenguaje PDDL, representación STRIPS de un problema, el proceso de
\emph{grounding}, la complejidad de encontrar un plan que resuelva la tarea, y
porque es necesario utilizar la aproximación de planes relajados.

Posteriormente, se profundizará en el área de \emph{machine learning} abarcando
\textcolor{red}{la representación de palabras y oraciones dadas por \emph{word
embeddings}}, conceptos de aprendizaje supervisado, modelos de clasificación, y
métricas.

Además, se irán introduciendo de manera progresiva de que manera estas dos áreas
se complementan para lograr nuestro objetivo de guiar el proceso de
\emph{grounding}.

\section{Planning}

\subsection{Intuición de Planning Clásico}

Durante nuestras actividades de la vida cotidiana, siempre estamos actuando y
anticipando el resultado de nuestras decisiones, aún si no estamos completamente
conciente de ello, o explícitamente planeando que hacer antes de realizar una
acción. No obstante, una tarea que abarque objetivos nuevos y complejos requiere
deliberación al consistir de acciones que uno no esta acostumbrado, tareas de
alto riesgo, o cooperación con alguien más.
\citep{Nau-Ghallab-Malik-Traverso-2004}

\emph{Automated Planning} es el area de la inteligencia artificial que estudia
la deliberación de los procesos computacionales. Su objetivo es facilitar la
tarea de planificación por medio del razonamiento sobre representaciones
abstractas y formales del dominio, configuración inicial, combinación de
acciones, y objetivos a ser satisfacidos. El modelo conceptual del dominio en el
cual las acciones son ejecutadas es llamado \emph{planning domain}, y los
requerimientos inicial y final son denominados \emph{init} y \emph{goal}. Estas
3 componentes definen un \emph{planning problem} cuyo objetivo es encontrar
alguna combinación de acciones que determinen un \emph{plan}.

Algunos ejemplos de esto son la navegación de un robot, donde una acción altera
su posición siendo necesarias aquellas que lo lleven de una habitación a otra;
en el caso de un micro-procesador, una instrucción puede pensarse como una
acción que cambia el valor de sus registros y se busca que instrucciones son
necesarias para ejecutar un programa; un servicio web de vuelos de avion que
reciba un mensaje de reservación por un vuelo, siendo una acción que cambia su
estado, siempre y cuando no se sobrepase el número de reservaciones máximo a
partir de una cantidad inicial disponible. \citep{Sandewall-2008-HandbookOK}

\subsection{Representación de acciones y estados}

Una tarea de \emph{planning} puede ser descripta en términos de un \emph{init},
\emph{goal}, y acciones que un agente puede llevar a cabo.

\paragraph{Representación de estados.} Cada estado posible de una tarea de
\emph{planning} se representa en términos de conjuntos de átomos o símbolos
proposicionales como $p, q, r$. Estos símbolos modela una situación dentro del
entorno. Por ejemplo, el símbolo $p$ podría representar "El satelite $s_{10}$
apunta al planeta $tierra$"


Dado que \emph{planning} es un proceso complejo y costoso, es solo requerida
cuando es estrictamente necesaria, o cuando el \emph{trade-off} costo beneficio
es realmente a favor. Además, debido a esta dificultad solo se necesitan planes
que sean factibles y relativamente buenos en lugar de planes óptimos.

\emph{Planning} como disciplina emerge de la investigación en áreas tales como
búsuqeda en un espacio de estados, demostración de teoremas, teoría de control,
y bajo las necesidades prácticas en robótica, control de misiones espaciales,
entre otros.

Alguno de los motivos de planificación automática es el diseño de herramientas
que den lugar a eficiente y asequibles recursos para la toma de decisiones. Un
ejemplo de ello es una operación de rescate luego de un desastre natural tal
como un terremoto o inundación. Dicha tarea necesita de una excelente
comunicación por parte de los rescatistas siendo indispensable la necesidad de
distintos planes.