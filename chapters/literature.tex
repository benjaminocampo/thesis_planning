\chapter{Fundamentos Teóricos}
\label{ch:lit_rev}

En esta sección se detallaran una serie de definiciones teóricas, necesarias
para comprender el enfoque de esta tesis. Partiendo desde el área de
\emph{planning}, detallando más concretamente que son los estados y acciones, el
lenguaje PDDL, representación STRIPS de un problema, el proceso de
\emph{grounding}, la complejidad de encontrar un plan que resuelva la tarea, y
porque es necesario utilizar la aproximación de planes relajados.

Posteriormente, se profundizará en el área de \emph{machine learning} abarcando
\textcolor{red}{la representación de palabras y oraciones dadas por \emph{word
embeddings}}, conceptos de aprendizaje supervisado, modelos de clasificación, y
métricas.

Además, se irán introduciendo progresivamente de que manera estas dos áreas
se complementan para lograr nuestro objetivo de guiar el proceso de
\emph{grounding}.

\section{Planning}

\subsection{Intuición de planning clásico}

Durante nuestras actividades de la vida cotidiana, siempre estamos actuando y
anticipando el resultado de nuestras decisiones, aún si no estamos completamente
conciente de ello, o explícitamente planeando que hacer antes de realizar una
acción. No obstante, una tarea que abarque objetivos nuevos y complejos requiere
deliberación al consistir de acciones que uno no esta acostumbrado, tareas de
alto riesgo, o cooperación con alguien más.
\citep{Nau-Ghallab-Malik-Traverso-2004}

\emph{Automated Planning} es el area de la inteligencia artificial que estudia
la deliberación de los procesos computacionales. Su objetivo es facilitar la
tarea de planificación por medio del razonamiento sobre representaciones
abstractas y formales del dominio, configuración inicial, combinación de
acciones, y objetivos a ser satisfacidos. El modelo conceptual del dominio en el
cual las acciones son ejecutadas es llamado \emph{planning domain}, y los
requerimientos inicial y final son denominados \emph{init} y \emph{goal}. Estas
3 componentes definen un \emph{planning problem} cuyo objetivo es encontrar
alguna combinación de acciones que determinen un \emph{plan}.

Algunos ejemplos de esto son la navegación de un robot, donde una acción altera
su posición siendo necesarias aquellas que lo lleven de una habitación a otra;
en el caso de un micro-procesador, una instrucción puede pensarse como una
acción que cambia el valor de sus registros y se busca que instrucciones son
necesarias para ejecutar un programa; un servicio web de vuelos de avion que
reciba un mensaje de reservación por un vuelo, siendo una acción que cambia su
estado, siempre y cuando no se sobrepase el número de reservaciones máximo a
partir de una cantidad inicial disponible. \citep{Sandewall-2008-HandbookOK}

\subsection{Estados y acciones}

Un mundo o estado  posible de una tarea de \emph{planning} se representa a
partir de un conjunto de símbolos proposicionales que modelan un aspecto del
entorno. Por ejemplo, el símbolo proposicional $p$ puede modelar la situación
``el agente $A_1$ se encuentra en la habitación $R_1$``. Además, aquellos
símbolos  que no están mencionados en un estado, se asumen como no válidos en
dicho mundo. De esta manera, si se tienen los símbolos proposicionales $p,q,r$,
$\{p, q\}$ representa el estado en el que vale $p$ y $q$ pero no $r$.

Las acciones son representadas en términos de una precondición y un efecto. La
precondición es una fórmula proposicional que representa la condición necesaria
para que la acción pueda ser llevada a cabo. Mientras que el efecto es una
fórmula que determina los cambios que produce la ejecución de la acción sobre el
entorno. Por ejemplo, consideremos la siguiente acción $A$:
\begin{align*}
    & Action : A \\
    & pre : p \land \neg r \\
    & eff : \neg p \land q
\end{align*}

$A$ no es aplicable en el estado $\{p, r\}$, ya que, no se cumple su
precondición, pero si en el estado $\{p\}$ transformandolo en el estado $\{q\}$. Se puede interpretar las acciones como operadores de transformación de
estados en el que su ejecución genera que ciertos símbolos proposicionales
empiecen a valer o no según si estos ocurren positivamente o negativamente en el
efecto de la acción.

El tipo de fórmula que se permite en la precondición y en el efecto de las
acciones, así como también el usado para describir la meta, determina el tipo de
la tarea de \emph{planning}. En particular, se utilizaron tareas de
\emph{planning} STRIPS durante el desarrolo de esta tesis.

\subsection{Tareas STRIPS}

El tipo de una tarea de \emph{planning} está dada por la lógica de las formulas
que ocurren en las acciones y en la meta. En STRIPS, las fórmulas son
conjunciones de literales, es decir, de la forma $\bigwedge_i l_i$ con $l_i$
un símbolo proposicional o su negación.

\begin{mydef}
    Una fórmula STRIPS es una fórmula $\phi$ tal que $\phi$ está en forma 1-CNF.
    Una acción a es del tipo STRIPS si su precondición y su efecto son fórmulas
    STRIPS. Una meta g es del tipo STRIPS si g es una fórmula STRIPS.
\end{mydef}

Una forma más conveniente de trabajar sobre esta representación es utilizando
únicamente conjuntos de símbolos proposicionales. Vimos que en el caso de los
estados, solo se mantienen aquellos que son válidos en el mundo, es decir, los
literales positivos. De manera similar, ocurre con la precondición de una
acción. Sin embargo, en el caso de su efecto se mantienen dos conjuntos, uno
con los símbolos proposicionales positivos, y otro con los negativos.

Por ejemplo, la acción $A$ que describimos anteriormente puede verse
de la siguiente forma:
\begin{align*}
    & Action : A \\
    & pre : \{ p \}\\
    & add : \{ q \}\\
    & del : \{ p \}
\end{align*}

La interpretación de $A$ es similar a la que se dió anteriormente, la
precondición contiene los símbolos proposicionales necesarios para que $A$ sea
aplicable en un mundo. Mientras que $add$ y $del$ son los conjuntos de símbolos
proposicionales que agrega y elimina la acción producto de su ejecución.

Esta nueva representación permite abstraer fórmulas y operadores
proposicionales, lo cual ayudará para introducir de manera más natural algunas
técnicas del área.

\begin{mydef}
    Una tarea de planning STRIPS es una 4-upla $\Pi = (F, A, I, G)$ donde $F$ es
    un conjunto finito de símbolos proposicionales denominados facts, $A$ es un
    conjunto finito de acciones STRIPS, $I \subseteq F$ el estado inicial, $G
    \subseteq F$ el estado final.
\end{mydef}

\begin{mydef}
    Sea $\Pi = (F, A, I, G)$ una tarea STRIPS.
    
    \begin{itemize}
        \item Un estado $s \subseteq F$ es un conjunto de facts. Diremos que un
        símbolo proposicional $p \in F$ vale en un estado $s$ sii $p \in s$.
        
        \item Una acción STRIPS es una 3-upla $a = (pre, add, del)$, tal que,
        pre, add, del, son subconjuntos de F, y los denotaremos como pre(a),
        del(a), y add(a) respectivamente.

        \item Una acción $a$ es aplicable en un estado $s$ si $pre(a) \subseteq
        s$, en tal caso, el estado resultante es $s' = (s \rvert del(a)) \cup
        add(a)$. Escribimos $s \rightarrow^a s'$ para la transición de $s$ a
        $s'$ vía $a$. Para una secuencia de acciones $\vec{a}$, escribimos $s
        \rightarrow^{\vec{a}} t$ si estos pueden ser iterativamente aplicados a
        $s$, resultando en $t$.

        \item Un plan para $\Pi$ es una secuencia $\vec{a}$ con $I
        \rightarrow^{\vec{a}} s_G$ si $G \subseteq s_G$.
        
        \item Una tarea $\Pi$ es satisfacible si un plan para $\Pi$ existe. El
        plan es denominado óptimo si es el que tiene longitud más corta de entre
        todos los planes para $\Pi$.
    \end{itemize}
\end{mydef}

Para ejemplificar las definiciones anteriores, consideremos el ...
\textcolor{blue}{TODO: Ver un ejemplo para este caso}

\subsection{Representaciones STRIPS}

Un dato de entrada necesario para cualquier algoritmo de planificación es una
descripción del problema a ser resuelto. En la práctica, es usualmente imposible
incluir una explicita enumeración de todos los posibles estados y transiciones
que se pueden realizar en el dominio a partir de una tarea STRIPS. Por lo tanto,
es necesario una representación que permita computarlas dinámicamente.

Supongamos que queremos modelar la ubicación de un agente $A_1$ y nuestro mundo
tiene $n$ ubicaciones. Entonces vamos a necesitar $n$ símbolos proposicionales
$p_i$ tal que representen la acción ``El agente $A_1$ está en la ubicación
$i$``. Si en lugar de un solo agente hubiesen una cantidad $m$ de ellos. Se
necesitarian $n \times m$ símbolos para modelar esta caracteristica de la tarea.
Una forma más compacta de representar esta propiedad es por medio de un
predicado de la forma $at(x, y)$ donde $x$ denota a un agente, e $y$ su
ubicación, siendo mucho más flexible debido a su naturaleza esquematica.

De manera similar, si se quisiera modelar una acción de cambio de posición,
haría falta describir una por cada agente y par de ubicaciones obteniendo un
total de $m \times n \times n$ instancias. Una mejor alternativa es representar
esta transformación por medio de un esquema de acción de la forma $move(x, s,
d)$ donde $x$ representa al agente, $s$ la ubicación actual, y $d$ la ubicación
a la cual moverse.

En síntesis resulta más adecuado para la especificación utilizar predicados y
esquemas de acción en lugar de símbolos proposicionales. Esto lleva a las
siguientes definiciones:

\begin{mydef}
    Una especificación de una tarea STRIPS es una 6-upla $(\mathcal{P},
    \mathcal{A}, \Sigma^{C}, \Sigma^{O}, I, G)$  donde $\mathcal{P}$ es un
    conjunto de predicados, $\mathcal{A}$ es un conjunto de esquemas
    de acción, $\Sigma^{C}$ es un conjunto de constantes, $\Sigma^{O}$ es un
    conjunto de objetos no constantes, I es el estado inicial, y G es la meta.
\end{mydef}

\begin{mydef}
    Sea $(\mathcal{P}, \mathcal{A}, \Sigma^{C}, \Sigma^{O}, I, G)$ una
    especificación de una tarea STRIPS.

    \begin{itemize}
        \item Un esquema de acción es una 3-upla $a[X] = (pre(a), add(a),
        del(a))$ donde cada elemento es subconjunto de $\mathcal{P}$, y $X$ es el
        conjunto de variables que ocurren en $pre(a) \cup add(a) \cup del(a)$ y
        que son parte de $\Sigma^{O}$.
        \item Un predicado es un símbolo atómico de $\mathcal{P}$ de la forma
        $p[X]$ con $X$ un conjunto de variables que ocurren en su interfaz y
        que son parte de $\Sigma^{O}$.
    \end{itemize}
\end{mydef}

\textcolor{blue}{TODO: ver otro ejemplo}

\subsection{Relajación por deletes}

El problema de decidir si una tarea de planning STRIPS tiene un plan es
PSPACE-completo, lo cual lo hace intratable para problemas complejos
\citep{Nau-Ghallab-Malik-Traverso-2004}. Sin embargo, en ocasiones se suelen
realizar simplificaciones para obtener uno nuevo, a priori más sencillo de
resolver, que aporte información para resolver el original. En particular, la
\emph{delete-relaxation} es una relajación de la tarea que consiste en eliminar
las precondiciones y poscondiciones negativas de las acciones. Esta relajación
es ampliamente utilizada en planning y su importancia subyace en la siguiente
propiedad: en la \emph{delete-relaxation} si un \emph{fact} se vuelve válido,
entonces permanecerá válido para siempre.

\begin{mydef}
    Sea $\Pi = (F, A, I, G)$ una tarea STRIPS.
    \begin{itemize}
        \item Sea $a \in A$, definimos a la acción relajada por delete $a^{+}$
        dada por $pre(a^{+}) = pre(a)$, $add(a^{+}) = add(a)$, y $del(a^{+}) =
        \emptyset$.

        \item Denotaremos con $A^{+} = \{a^{+} : a \in A\}$ al conjunto de
        acciones relajadas por delete.

        \item Para una secuencia de acciónes $\vec{a}$ denotaremos con
        $\vec{a}^{+}$ a la secuencia de acciones relajadas por delete.

        \item Denotaremos con $\Pi^{+} = (F, A^{+}, I, G)$ a la tarea STRIPS
        relajada por deletes.

        \item Un plan relajado para $\Pi$ es un plan para $\Pi^{+}$.
    \end{itemize}
\end{mydef}

Algo a destacar de esta simplificación es que aligera las restricciones que una
acción puede tener. Por ejemplo, en el problema del agente, si este se desplazo
por varias ubicaciones, tendriamos que se encuentra en más de un lugar al mismo
tiempo ya que la proposición que indicaba su ubicación anterior no puede ser
borrada.

\subsection{Proceso de grounding}

Dada una especificación de una tarea STRIPS $(\mathcal{P}, \mathcal{A},
\Sigma^{C}, \Sigma^{O}, I, G)$ se puede obtener su correspondiente tarea $\Pi =
(F, A, I, G)$ recolectando todas las instancias posibles de predicados en
$\mathcal{P}$ y esquemas de acción de $\mathcal{A}$ con objetos de $\Sigma^{C}$.
Es decir, $F$ contiene un \emph{fact} por cada posible asignación de objetos a
los argumentos de cada predicado $P[X] \in \mathcal{P}$, y $A$ contiene una
acción por cada posible asignación de objetos a los argumentos de cada esquema
de acción $a[X] \in \mathcal{A}$. Más formalmente,



En la sección 2.1.3 se hizo mención a que es una tarea STRIPS para luego
detallar como se especifan en la sección 2.1.4. 


\textcolor{blue}{TODO: Lo que está de acá en adelante capaz puede ir en algún lado?}

Dado que \emph{planning} es un proceso complejo y costoso, es solo requerida
cuando es estrictamente necesaria, o cuando el \emph{trade-off} costo beneficio
es realmente a favor. Además, debido a esta dificultad solo se necesitan planes
que sean factibles y relativamente buenos en lugar de planes óptimos.

\emph{Planning} como disciplina emerge de la investigación en áreas tales como
búsuqeda en un espacio de estados, demostración de teoremas, teoría de control,
y bajo las necesidades prácticas en robótica, control de misiones espaciales,
entre otros.

Alguno de los motivos de planificación automática es el diseño de herramientas
que den lugar a eficiente y asequibles recursos para la toma de decisiones. Un
ejemplo de ello es una operación de rescate luego de un desastre natural tal
como un terremoto o inundación. Dicha tarea necesita de una excelente
comunicación por parte de los rescatistas siendo indispensable la necesidad de
distintos planes.