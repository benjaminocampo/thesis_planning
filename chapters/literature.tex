\chapter{Fundamentos Teóricos}
\label{ch:lit_rev} %Label of the chapter lit rev. The key ``ch:lit_rev'' can be used with command \ref{ch:lit_rev} to refer this Chapter.

En esta sección se detallaran una serie de definiciones teóricas, necesarias para comprender el enfoque de esta tesis. Partiendo desde el área de \emph{planning}, detallando más concretamente que son los estados y acciones, el lenguaje PDDL, representación STRIPS de un problema, la complejidad de encontrar un plan que resuelva la tarea, y porque es necesario utilizar la aproximación de planes relajados.

Posteriormente, se profundizará en el área de \emph{machine learning} abarcando \textcolor{red}{la representación de palabras y oraciones dadas por \emph{word embeddings}}, conceptos de aprendizaje supervisado, modelos de clasificación, y métricas.

Además, se irán introduciendo de manera progresiva de que manera estas dos áreas se complementan para lograr nuestro objetivo de guiar el proceso de \emph{grounding}.

% PLEAE CHANGE THE TITLE of this section
\section{Conceptos Básicos de Planning Clásico}
Planning es el 