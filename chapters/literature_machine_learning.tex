\chapter{Aprendizaje automático}

En este capítulo se profundizará en el área de \emph{aprendizaje automático}
abarcando la representación de palabras y oraciones dadas por one-hot encoders,
word embeddings, conceptos de aprendizaje supervisado y no supervisado, modelos
de clasificación, y métricas.

\section{Orígenes y evolución}

El aprendizaje automático es el campo de la inteligencia artificial que busca
desarrollar programas que aumáticamente mejoren en base a la experiencia. Estos
métodos difieren del típico paradigma de implementación donde en lugar de ser
``programado`` es ``entrenado``. Tal entrenamiento consiste en disponerle al
algoritmo información en forma de ejemplos con el fin de reconocer patrones
estadísticios y eventualmente determinar reglas que sirvan para automatizar una
tarea.

En los últimos años varias aplicaciones se han beneficiado de esta área, desde
programas relacionados con la detección fraudulenta de transacciones con tarjeta
de credito, sistemas de recomendación que guían usuarios en un servicio de
acuerdo a sus preferencias, o incluso vehículos que se manejan sin necesidad de
la intervención del conductor. Al mismo tiempo, una importante cantidad de
avances teóricos y algorítmicos se fueron realizando formando las bases de este
campo. TODO: Poner referencias a los ejemplos.

Aún con estos increibles logros, no se conoce aún como crear computadoras que
aprendan al nivel que las personas lo hacen. No obstante, se han desarrollado
algoritmos que se han aproximado a este objetivo siendo efectivos para varios
tipos de problemas.

\section{Tipos de aprendizaje}

\subsection{Aprendizaje supervisado}

El aprendizaje supervisado es una subárea del aprendizaje automático cuyo
objetivo es deducir una modelo a partir de datos anotados y permita mapear
ejemplos no vistos previamente. Esta información está compuesta por un conjunto
de ejemplares y sus correspondientes etiquetas, en ocasiones dada por un
anotador humano, la cual indica el resultado del modelo a partir del dato como
entrada. Las etiquetas pueden ser categóricas o continuas determinando un
problema de clasificación o regresión. Algunas tareas más frecuentes de
clasificación son la categorización de documentos, reconocimiento de lenguaje
ofensivo, o análisis de sentimiento. Mientras que para regresión, lo son las
estimaciones de precios de artículos, objetos, o viviendas.

El resultado de ejecutar un algoritmo de machine learning supervisado se puede
expresar como una función $f(\vect{x})$ que recibe un ejemplar $\vect{x}$ como
entrada y genera un vector $\vect{y}$ de salida codificado de la misma manera
que las etiquetas. Las muestras utilizadas para ajustar $f$ está dado por
vectores $\{\vect{x}_1, ..., \vect{x}_n\}$ y anotaciones $\{y_1, .. y_n\}$
conformando el \emph{conjunto de entrenamiento}.

La forma precisa de $f$ es determinada durante la fase de entrenamiento. Una vez
transcurrida, se puede estimar la identidad de nuevos datos etiquetados
pertenecientes al \emph{conjunto de test} con el fin de evaluar la performance
del modelo. En el caso que la predicción se aproxime a la esperada para estas
nuevas entradas, el modelo logró generalizar la tarea.

Sin embargo, los modelos de aprendizaje supervisado están limitados por datos
anotados disponibles y su disponibilidad dependerá de la dificultad para
obtenerlos. Algunas tareas son relativamente sencillas para etiquetar para
cualquier persona, (e.g. como determinar si en una foto aparece un gato),
mientras que en otros puede llegar a requerir humanos que sean expertos de
dominio (e.g. abogados, médicos, linguistas, etc).

\subsection{Aprendizaje no supervisado}

En contraste al método anterior, el aprendizaje no supervisado consiste en
descubrir automáticamente patrones sobre los datos de entrenamiento que permitan
explicarlos sin depender de datos de anotados. Es considerada como una parte del
área del análisis y exploración. Es por eso que no pueden ser evaluados
basándose en exactitud o precisión, si no más bien en la cantidad de información
que podamos extraer de los datos a partir del uso de estas técnicas. Algo a
considerar es lo sencillo que es disponer de estos datos al no requerir ningún
tipo de supervición humana.

\section{Codificación de características}

Los modelos de aprendizaje automático dependen fuertemente de como se
representen los datos de entrada que uno tiene disponible. Preparar los
ejemplares para acoplarse apropiadamente a un algoritmo de machine learning con
el fin de mejorar la performance del modelo es una de las tareas las cuales los
científicos de datos disponen la mayor parte de su atención y tiempo. Alguna de
las obligaciones que incluye son la imputación de valores faltantes, manejo de
outliers, estandarización y escalado, \emph{binning}, y codificación.

En particular pondremos nuestro foco en la codificación de características
desarrollando métodos \emph{one-hot} y \emph{word embeddings (word embeddings)}.

\subsection{One-hot encoding}

\subsection{Word embeddings (vectores de palabras)}

\section{Entrenamiento y evaluación}

\subsection{Validación cruzada}

\subsection{Búsqueda de hiperparámetros}

\section{Algoritmos de clasificación}

\subsection{Regresión logistica}

\subsection{XGBoost}

\subsection{Redes neuronales}

\section{Métricas de clasificación}